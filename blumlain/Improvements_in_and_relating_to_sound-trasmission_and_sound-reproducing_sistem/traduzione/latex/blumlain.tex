\documentclass[11pt]{article}
\usepackage{geometry} 
\geometry{a4paper}   
\usepackage{multicol}
\usepackage[english, italian]{babel}
\usepackage{fancyhdr}
\usepackage{lettrine}
\usepackage{musicography}
\geometry{centering}


\pagestyle{fancy} 
\lhead{\textsf{Traduzione di Gabriele Petrillo}}                                %serve ad inserire la linea sopra e il titolino
\rhead{\textsf{Brevetto n° 394,325}}
\renewcommand{\headrulewidth}{5pt} %grandezza della linea in alto
\renewcommand{\footrulewidth}{1pt}   % grandezza della linea in basso


\begin{document}

\begin{minipage}{0.9\linewidth}
\vspace{0.3cm}
\begin{center}

Data richiesta: 14 - 12 - 1931. N°. 34.657/31\\\
Complete Left: 10 - 11 - 1932\\\
Complete Accepted: 14 - 6 - 1933.\\\

DESCRIZIONE COMPLETA

\end{center}

\end{minipage}

\vspace{1cm}

\begin{minipage}{0.95\linewidth}
\begin{center}
{\Large{\textbf{\textsf{Miglioramenti relativi ai sistemi di trasmissione, registrazione e riproduzione del suono.}}}} \\

\end{center}
\end{minipage}
\vspace*{0.8cm}


%=========FOREWORD========================

\begin{multicols*}{2}
\setlength{\parskip}{0pt}

Noi,  \textsc{Alan Dower Blumlein}, di 57 anni, Earl’s Court Square, London, S.W.5, cittadino Britannico, e \textsc{Electric and Musical Industries, Limited}, di Blyth Road, Hayes, nella contea di Middlesex, una compagnia registrata sotto le leggi della Gran Bretagna, con la presente dichiarano la natura di questa invenzione e in che modo deve essere eseguita, sarà quindi descritta e verificata.

Questa invenzione riguarda la trasmissione, registrazione e riproduzione del suono ed è rivolta in modo particolare ai sistemi di registrazione e riproduzione di parlato, musica e altri effetti sonori. É applicabile in particolare, anche se non in modo esclusivo, ad i sistemi associati alle immagini come il cinema.

Lo scopo fondamentale di questa invenzione è fornire un sistema di registrazione, riproduzione e/o trasmissione in grado di trasmettere, attraverso due percorsi acustici, un’impressione realistica dell’esperienza uditiva e questo riguarda anche l’idea di trasmettere un impressione reale della direzione e quindi, nel caso in cui il suono sia associato alle immagini, aumentare l’illusione che il suono provenga, e provenga solo, dall’attore o da un altra sorgente sonora presentata visivamente.

L’invenzione non è, tuttavia, limitata all’uso per il cinema, ma, per esempio, può essere usata per migliorare la qualità degli annunci pubblici, sistemi di trasmissione telefonici o radiofonici, o per migliorare la qualità delle registrazioni audio. Quando si  registra musica si riscontrano molti problemi con lo spiacevole effetto prodotto dall’eco che in una situazione normale non verrebbe notato da nessuno che stia ascoltando nella stanza in cui si sta svolgendo la performance. Un osservatore nella stanza ascolta con il significato direzionale associato alla musica eseguita nella stanza. Pertanto questi echi vengono ignorati e l’attenzione viene focalizzata sulla fonte sonora. Quando la musica è riprodotta attraverso un singolo canale gli echi arrivano dalla stessa direzione del suono diretto creando confusione. Lo scopo secondario di questa invenzione  è dare un significato direzionale al suono in modo che quando venga riprodotto gli echi vengano percepiti come tali.

Per capire le basi fisiche alla base dell’invenzione e i suoi sviluppi, saranno brevemente riassunte le relazioni fisiche tra le sorgenti sonore, l’emissione delle onde sonore e le orecchie.

L’abilità di determinare la direzione da cui arriva un suono è derivato dall’ascolto binaurale, il cervello è in grado di determinare la differenza tra i suoni ricevuti da una sorgente alle due orecchie e quindi determinare la direzione angolare da cui i vari suoni arrivano. Questa funzione è ben conosciuta ed è stata molto usata per esempio nel rilevamento della direzione sub-acquea in cui due microfoni sono connessi alle cuffie, uno per orecchio dell’osservatore, i due canali tra i microfoni e le orecchie sono completamente separati.

Con due microfoni correttamente spaziati e i due canali tenuti completamente separati, per esempio usando le cuffie, si può ottenere questo effetto direzionale per esempio in uno studio. Se, tuttavia, i canali non sono completamente separati (come, ad esempio nel caso delle disposizioni proposte precedentemente per la registrazione e/o riproduzione del suono, in cui i suoni sono rilevati da molti microfoni omnidirezionali e portati ad altoparlanti che prendono il posto delle cuffie) l’effetto è quasi completamente perso e questi sistemi non sono diventati di uso comune in quanto sono decisamente insoddisfacenti allo scopo. La presente invenzione tiene conto del controllo del suono, emesso per esempio da tali altoparlanti, in modo da mantenere l’effetto direzionale.

Non è ancora del tutto chiaro come le orecchie siano in grado di determinare la direzione di una sorgente sonora tuttavia è abbastanza noto che i principali fattori interessati sono le differenze di fase e le differenze di intensità tra i suoni che raggiungono le due orecchie, l’influenza che ha ciascuno di questi dipende dalla frequenza del suono emesso. Per le onde a bassa frequenza c’è una piccola o nessuna differenza di intensità alle due orecchie ma c’è una forte differenza di fase. Per una data direzione del suono la differenza di fase è approssimativamente proporzionale alla frequenza, rappresentando un ritardo fisso tra il suono che arriva alle due orecchie, in base a questo ritardo il cervello decide la direzione da cui proviene il suono. Questa operazione vale per tutte le frequenze fino a quella per cui c’è una differenza di fase di π radianti o più tra i suoni che arrivano alle due orecchie da una sorgente situata sulla linea che le unisce; ma al di sopra di tale frequenza se la differenza di fase fosse la sola caratteristica su cui si faceva affidamento per la localizzazione direzionale ci sarebbe stata un ambiguità nella posizione apparente della sorgente. In quel momento comunque la testa inizia a funzionare da divisorio e causa notevoli differenze di intensità tra il suono che raggiunge le due orecchie, ed è valutando questa differenza di intensità che il cervello determina la direzione del suono alle alte frequenze. La soglia frequenziale a cui il cervello cambia dalla discriminazione di fase a quella d’intensità è intorno ai 700 c.p.s. ma si deve essere chiaro che questo può variare entro limiti abbastanza ampi in circostanze diverse e da persona a persona, e che in ogni caso il trasferimento non è improvviso o discontinuo ma c’è una considerevole sovrapposizione dei due fenomeni in modo tale che in un intervallo di frequenza considerevole le differenze sia di fase che di intensità avranno in qualche misura un effetto nel determinare il senso di direzionalità dei suoni.

Da queste considerazioni è chiaro che un effetto direzionale è ottenuto fornendo alle due orecchie l’impressione di differenze di fase per le basse frequenze e d’intensità per le alte frequenze e potrebbe sembrare che sia possibile ottenere questo effetto riproducendo da due altoparlanti la differenza ricevuta da due microfoni spaziati in modo da rappresentare le orecchie umane e collegati ognuno ad un solo altoparlante. Si può dimostrare tuttavia che le differenze di fase necessarie alle orecchie per la sensazione direzionale delle basse frequenze non sono prodotte solo dalle differenze di fase dei due altoparlanti (entrambi comunicano con entrambe le orecchie) ma che per dare un effetto di differenza di fase è necessaria anche la differenza di intensità tra gli altoparlanti; mentre per le alte frequenze le differenze di intensità iniziali dalla sorgente non sono sufficientemente marcate quando il suono raggiunge le orecchie, e quindi, per produrre effetti adeguati, le differenze di intensità devono essere amplificate. È per questo motivo che i metodi proposti prima (dove sono usati solo microfoni omnidirezionali) non sono sufficienti per ottenere l’effetto desiderato, questi accorgimenti necessari non sono stati capiti o ottenuti in nessun modo nelle disposizioni precedenti. 

Si vedrà quindi che l’invenzione consiste sostanzialmente nel controllare in questo modo l’intensità del suono che deve essere emesso da una pluralità di altoparlanti o fonti sonore simili, in una relazione spaziata adeguata all’ascoltatore, in questo modo le orecchie dell’ascoltatore noteranno la differenza di fase alle basse frequenze adatte per dare al cervello il desiderato effetto di direzione del suono di origine. In altre parole la direzione dal quale il suono arriva ai microfoni determina le caratteristiche (in modo particolare, come sarà evidente in seguito, le intensità) dei suoni emessi dagli altoparlanti in modo da fornire questa sensazione direzionale.

Deve essere chiaro che il controllo manuale da parte di un operatore dell’intensità di una pluralità di altoparlanti spaziati attorno un schermo è stato già proposto ma questo metodo soffre considerevolmente dei difetti sopra indicati, ed in ogni caso è molto difficile e scomodo da usare. Non è stata dichiarata alcuna novità per il semplice controllo dell’intensità, eccetto nella misura in cui la natura del controllo è tale da fornire la necessaria differenza di sensazione relativa alla differenza di fase ed intensità.

Se, come da questa invenzione, il suono è prima registrato e successivamente riprodotto, il controllo può essere completamente effettuato sia durante la registrazione che durante la riproduzione, o può essere parzialmente eseguito in ciascuna  delle due fasi. In questa specifica le parole “trasmissione del suono” (più in particolare nelle affermazione di seguito specificate) sono utilizzate (a meno che il contesto non richieda diversamente) non solo nel caso in cui l’impulso passa direttamente dal microfono all’altoparlante, ma anche in tutte quelle disposizioni che comprendono un processo o un sistema di registrazione intermedio; e in questi ultimi casi questa definizione si applica al passaggio di impulsi sia dai microfoni al sistema di registrazione, che dal riproduttore agli altoparlanti.

Più specificatamente l’invenzione consiste in un sistema di trasmissione del suono in cui lo stesso è raccolto da una pluralità di capsule microfoniche e riprodotta da una pluralità di altoparlanti, comprendenti due o più microfoni direzionali e/o un sistema di elementi nel circuito di trasmissione o circuiti in base ai quali l’intensità relativa agli altoparlanti è resa dipendente dalla direzione da cui i suoni arrivano ai microfoni.

L’invenzione consiste anche in un sistema di trasmissione del suono ricevuto da due o più microfoni, in cui la differenza di fase delle pressioni sonore alle basse frequenze ricevuta dai microfoni è riprodotta come differenza d’intensità degli altoparlanti.

L’invenzione consiste inoltre in un sistema di trasmissione del suono in cui il suono originale viene rilevato da due o più microfoni direzionali la cui sensibilità varia con la direzione del suono incidente, e in cui la dipendenza delle risposte relative dei microfoni dalla direzione di un’onda sonora incidente viene utilizzata per controllare i volumi relativi del suono emesso da due o più altoparlanti.

L’invenzione consiste anche in un sistema di trasmissione del suono in cui gli impulsi di due microfoni trasmessi su singoli canali sono adattati per interagire, sono quindi trasmessi altre due serie di impulsi consistenti rispettivamente nella metà della somma e metà della differenza degli impulsi originali, i suddetti impulsi sono in seguito modificati per controllare il volume relativo degli altoparlanti da cui il suono deve essere riprodotto. 

L’invenzione consiste anche in sistema di trasmissione del suono in cui il suono viene raccolto da due microfoni direzionali che sono così spaziati e/o con i loro assi di massima sensibilità così diretti l’uno all’altro e la sorgente sonora, che il volume relativo degli altoparlanti che riproducono gli impulsi sono controllati dalla direzione da cui il suono raggiunge i microfoni.

L’invenzione consiste anche in un sistema come indicato sopra in cui due serie di impulsi sono meccanicamente registrati nello stesso solco.

L’invenzione consiste anche in un sistema come indicato sopra in cui gli impulsi sono trasmessi da radiotelefonia.

L’invenzione consiste anche in un sistema come sopra esposto in combinazione con mezzi per la registrazione o la trasmissione di immagini.

La parola canale, come qui impiegata, indica un circuito elettrico che trasporta una corrente che ha una forma definita analoga ai suoni originali nello studio. Quindi i due canali sono differenti non solo perché l’intensità media o il tipo di corrente è differente ma alche perché sono originati da due microfoni in una posizione differente nello studio.

La natura dell’invenzione sarà evidente dalla seguente descrizione dei vari metodi e modi per attuarla tuttavia le differenti forme descritte sono fornite solamente a scopo di esempio e non impongono nessuna restrizione all’utilizzo dell’invenzione o il modo e i mezzi con cui può essere realizzata.

La descrizione sarà più chiara facendo riferimento ai disegni allegati:

Figura 1: rappresenta schematicamente l’assemblaggio di un sistema relativo all’invenzione;

Figura 2: rappresenta una disposizione microfonica per l’uso secondo una forma dell’invenzione;

Figura 3: rappresenta un sistema di trasformazione per una forma dell’invenzione; 

Figura 4: mostra una rappresentazione simbolica del sistema mostrato in figura 3;

Figure 8, 9, 10 e 11 rappresentano diverse forme di registratori che possono essere impiegati.

Figures 8, 9, 10 and 11 represent different forms of sound recorders which may be employed.

L’invenzione è particolarmente applicabile ai film e pertanto verrà data la seguente descrizione con riferimento a questa applicazione. In una forma dell’invenzione adatta a questo scopo mostrato in Fig.1, i suoni da registrare e riprodurre con le immagini possono essere ricevuti da una sorgente sonora\textit{a} da due microfoni omnidirezionali \textit{a1, a2} montati sui lati opposti di un blocco di legno o un paravento \textit{b} che servono a fornire differenza di intensità alle alte frequenze nello stesso modo in cui la testa umana opera sulle orecchie come indicato sopra. Le uscite dei due microfoni sono dopo un amplificazione separata con due amplificatori simili\textit{b1, b2} tmandati in un circuito \textit{c} opportunatamente organizzato comprendente trasformatori o ponti o reti di circuiti che convertono i due canali principali in due canali secondari che possono essere chiamati canali di somma e differenza. Questi sono disposti in modo tale che la corrente che scorre nel canale somma rappresenterà la metà della somma, o la media, della corrente che scorre nei due canali originali, mentre la corrente che scorre nel canale differenza rappresenterà la metà della differenza della corrente nei canali originali
	
Una disposizione dei trasformatori adatta a questo scopo è mostrata in Fig.3 dove gli ingressi di corrente degli amplificatori B1, B2 sono alimentati separatamente ciascuno a due avvolgimenti primari, uno su ciascuno dei due trasformatori, l’avvolgimento secondario di ciascun trasformatore fornisce una corrente in uscita di “somma” o “differenza”  a causa della direzione di avvolgimento delle bobine come mostrato. Una rappresentazione schematica di una disposizione di somma e differenza (che potrebbe consistere in un trasformatore simile a quello in Fig.3 o qualsiasi altra disposizione adatta di circuiti) è mostrato in Fig.4.

Relativamente alla forma dell’invenzione descritta, le due uscite dalla disposizione somma e differenza sono modificate per ottenere successivamente gli effetti sonori desiderati e una disposizione di un circuito per effettuare ciò è mostrato in Fig.5 che rappresenta la porzione dei circuiti indicati da \textit{c} in Figure 1. Supponendo che la corrente originale differisca solo in fase, la corrente nel canale differenza avrà una differenza in fase di  $\frac{\pi}{2}$ rispetto a quella nel canale somma. Questa differenza di corrente viene fatta passare attraverso due resistenze in serie \textit{d} ed \textit{e} tra le quali un condensatore \textit{f} forma un deviatore di corrente(shunt arm). l voltaggio attraverso questo condensatore F sarà in fase con quello nel canale somma. Passando la corrente nel canale somma attraverso una semplice rete di attenuatori resistivi composto dalle resistenze in serie \textit{g, h} e una una resistenza shunt \textit{i}, si ottiene una tensione che rimane in fase con la tensione che attraversa il condensatore \textit{f} nel canale differenza. Queste due tensioni vengono quindi combinate e separate di nuovo da un processo di somma e differenza come quello adottato precedentemente in modo da produrre due canali finali. Il voltaggio nel primo canale finale sarà la somma di questi voltaggi mentre nel secondo sarà la differenza tra questi voltaggi. Poiché queste tensioni erano in fase i due canali finali saranno in fase ma differiranno in ampiezza. Per una data frequenza, scegliendo il valore della resistenza shunt \textit{i} nel canale somma e del condensatore shunt \textit{f} è possibile ottenere qualsiasi grado di differenza di ampiezza in uscita relativa alla differenza di fase nei canali originali. Per le basse frequenze può essere dimostrato che la differenza di fase tra le onde, per una data direzione della sorgente sonora, varierà proporzionalmente alla frequenza, essendo molto piccola per una frequenza molto bassa. Quindi per una data direzione del suono la corrente nel canale differenza sarà maggiore rispetto a quella nel canale somma maggiore è la frequenza. Quindi l’uso di un condensatore shunt \textit{f} nel circuito differenza avrà l’effetto di produrre una differenza d’intensità fissa nei canali finali per una data direzione a tutte le basse frequenze.

Per le alte frequenze come indicato sopra non è necessario convertire gli sfasamenti in differenze di ampiezza. Il condensatore shunt \textit{f} nel circuito differenza è quindi costruito con una resistenza \textit{k} il cui valore è sostanzialmente uguale a quello della resistenza \textit{i}. 

Nella costruzione di questo circuito la capacità del condensatore \textit{f} è di tale valore che la sua impedenza è piccola rispetto a quella delle resistenze in serie \textit{d} e \textit{c} dell'intero raggio d’azione, mentre il valore della resistenza \textit{k} è tale da eguagliare la reattanza del condensatore approssimativamente alla frequenza al di sopra della quale si desidera non convertire le differenze di fase in differenze di ampiezza. Il valore di \textit{k} è in generale uguale a quello di i, in questo caso le differenze di ampiezza per le alte frequenze vengono trasmesse senza modifiche.

Può essere necessario impiegare circuiti più complessi della resistenza shunt \textit{k} e del condensatore \textit{f}nei circuiti di differenza e della resistenza shunt \textit{i} nel circuito di somma, che tuttavia formano la disposizione di base. Tuttavia, deve essere chiaro che i circuiti impiegati possono essere considerevolmente modificati come richiesto senza allontanarsi dall'ambito dell’invenzione.

Le uscite dal circuito di modifica \textit{c} (Figure 1) vengono passate agli amplificatori \textit{d1, d2} e quindi agli altoparlanti \textit{e1, e2} opportunamente disposti su ciascun lato di uno schermo. Resta inteso che la Figura 1 traccia semplicemente il passaggio dell'informazione dalla sorgente \textit{a} a un destinatario e non è stato mostrato alcun sistema di registrazione o riproduzione. Ciò può tuttavia essere inserito ovunque lungo il circuito elettrico, ad esempio tra gli amplificatori \textit{b1, b2} e modificando il gruppo \textit{c}, o tra il complesso \textit{c} e gli amplificatori \textit{d1, d2}. 

In quest’ultimo caso, gli impulsi trasmessi attraverso i due canali come indicato sopra possono ad esempio essere registrati su due tracce sonore su un film con qualsiasi mezzo idoneo o noto, ciascuna delle registrazioni possono comprendere una traccia sonora di densità costante e larghezza variabile (ad esempio una registrazione oscillografica). In alternativa, entrambe le registrazioni possono essere realizzate su un’unica traccia comprendente una combinazione delle forme di registrazione a larghezza variabile e densità variabile.

Tale registrazione può essere riprodotta facendo passare la luce dalla stessa fenditura attraverso le due tracce, separando il raggio nelle due porzioni registrate per mezzo di prismi o come mezzi ottici e impiegando le uscite da due celle fotoelettriche, eccitate da queste parti separate del raggio (dopo l’amplificazione) per far funzionare due altoparlanti disposti uno su ciascun lato dello schermo su cui sono proiettate le immagini del cinematografo.

Dalla descrizione di cui sopra sarà chiaro che l'obliquità della direzione di propagazione dell'onda sonora rispetto ai microfoni  \textit{a1, a2} produrrà differenze di intensità nei diffusori in modo da dare un'impressione ad un osservatore dell’incidenza del suono.

Se vengono utilizzati due microfoni molto piccoli e posizionati molto vicini tra loro, è possibile ottenere uscite microfoniche che non differiscono sensibilmente in ampiezza ma solo in fase per tutte le frequenze. In questo caso, il circuito di modifica può essere predisposto per convertire le differenze di fase in differenze di ampiezza nell'intero range di frequenze. Le differenze di fase trattate alle basse frequenze possono tuttavia essere così piccole che in questo caso lievi differenze nei due circuiti del microfono avrebbero effetti molto grandi. Per questo motivo è più adatta la spaziatura del microfono dello stesso ordine di quella delle orecchie umane.

Si comprenderà che la quantità di modifiche necessarie agli impulsi trasmessi attraverso i canali di somma e differenza come sopra indicato dipende da un numero di fattori, tra cui la spaziatura relativa dei microfoni e degli altoparlanti, nonché la dimensione e il posizionamento del schermo. Si può dimostrare che per le basse frequenze w il grado di modifica richiesto nel canale di differenza rispetto alla modifica nel canale di somma è dato da:
\[
K = \frac{2\nu}{j\omega} * \frac{y}{\theta k} * \frac{s}{x}
\]

Dove
\begin{description}
	\item [$\nu$] = velocità del suono.
	\item[y] = frazione della metà della larghezza del film per cui l’immagine della sorgente sonora è decentrata.
	\item[$\omega$] = angolo di obliquità, in radianti, della sorgente dal piano medio tra i microfoni.
	\item[k] = distanza effettiva tra i microfoni.
	\item[s] = larghezza dello schermo del teatro.
	\item[x] = distanza tra gli altoparlanti del teatro.
\end{description}

Questa espressione in effetti fornisce l'impedenza del condensatore shunt \textit{f} nel canale di differenza in termini della resistenza \textit{i} nel canale di somma. Vale per tutte le frequenze in cui \textit{k} è piccolo rispetto alla lunghezza d'onda e si basa sul presupposto che $\theta$ è piccolo e che \textit{x} e \textit{s} sono piccoli rispetto alla distanza dell'ascoltatore dallo schermo e dagli altoparlanti.

La parte $\frac{y}{\theta k}$ è un fattore della registrazione ed è costante per una determinata disposizione se la fotocamera è in linea con i microfoni e il centro dell'immagine, oppure l'azione non si sposta in modo sensibile da o verso i microfoni e la fotocamera. Durante la registrazione, le distanze relative di fotocamera e microfoni e la lunghezza focale dell'obiettivo possono essere regolate per mantenere costante questo fattore.

L'espressione $\frac{s}{x}$ è una costante per il teatro. Per quanto riguarda solo le basse frequenze, la distanza tra gli altoparlanti non deve superare la larghezza dello schermo, ma certamente non deve essere inferiore al 70\% della larghezza dello schermo. Più vicini sono gli altoparlanti, maggiore è la potenza necessaria per la gestione, ma minori sono i problemi introdotti dalla formazione di onde stazionarie.

Per le alte frequenze non esiste nessuna espressione definita e la modifica eventualmente utilizzata dovrà essere misurata empiricamente da prove ed errori.

Gli argomenti e la formula sopra riportati si basano su un'analisi delle onde dirette e potrebbe dover essere notevolmente modificata per consentire la riflessione di altri effetti acustici. É preferibile quindi impiegare il sistema a teatro in quanto entrano in considerazione tutti i fattori. È chiaro che, come indicato sopra, le reti di modifica e la disposizione dei canali possono essere implementati tra i microfoni e il film durante la riproduzione, e quest'ultimo soluzione, oltre a consentire la regolazione delle disposizioni per adattarsi al particolare teatro come indicato sopra, ha l'ulteriore vantaggio che il film può essere riprodotto da una singola testa o canale riproducente se, ad esempio, si rompe uno dei doppi arrangiamenti o in un teatro che, avendo un'installazione, non vuole installare un secondo apparato.

Per utilizzare con successo un sistema del tipo sopra descritto, è necessario effettuare esperimenti preliminari per determinare i valore di modifica più adatti da utilizzare per ciascuna registrazione ed è anche necessario standardizzare vari fattori della registrazione. Negli esperimenti preliminari, prima della registrazione, è possibile effettuare misurazioni dell'indicatore di volume con una sorgente sonora standard posta agli estremi del "set", ovvero lo spazio entro il quale deve essere effettuata la registrazione, e da queste proporre una rete di modifica. Un ulteriore esperimento può anche essere effettuato per standardizzare gli angoli di fase sul film. A teatro per controllare e bilanciare l’ingresso dei due canali può bastare una semplice regolazione e a tale scopo può essere usato un film di prova. Si vedrà quindi che l'attrezzatura teatrale totale necessaria è molto semplice e consiste in un modificatore di trasmissione (comprendente per esempio due o quattro trasformatori, resistenze di linea artificiali e la rete di controllo, che può essere non più di un condensatore e una resistenza) e due normali testine di riproduzione del suono o pick-up adattati per separare le due registrazioni su due canali di riproduzione completi. Non vi è alcun motivo per cui il secondo canale utilizzato non dovrebbe essere il canale "stand-by" installato spesso per motivi di sicurezza poiché se, come indicato sopra, uno dei canali interrompe la riproduzione può essere continuato senza gravi conseguenze solo sull'altro canale.

In relazione con la standardizzazione sopra indicata, mentre la frequenza di "trasferimento" binaurale (dalla fase alla discriminazione di intensità) non ha bisogno di un significato definito nelle registrazioni, poiché è una funzione del cervello umano, è comunque necessario definire un cambiamento relativo alla frequenza dalle alte alle basse frequenze durante la registrazione, poiché questa frequenza fissa i valori degli elementi nel modificatore e quindi la forma di modifica da utilizzare, la distanza tra i microfoni e la forma della paratia tra di loro. Qualunque frequenza conveniente può essere scelta come standard dopo che l'esperienza ha deciso quale sia la più adatta. Invece di una standardizzare è possibile intervenire elettricamente con esperimenti preliminari per la variazione delle posizioni dei microfoni e/o della spaziatura dei microfoni (sebbene quest'ultimo sia estremamente difficile) e si deve comprendere che questa disposizione rientra nell'ambito dell’invenzione.

L'analisi di cui sopra si basa su considerazioni che non tengono conto delle riflessioni del suono o delle interferenze durante la riproduzione. Le onde sonore riflesse che si presentano durante la registrazione saranno riprodotte con un senso direzionale e suoneranno più naturali di quanto farebbero con un sistema di riproduzione non direzionale. Se sorgono difficoltà nella riproduzione, possono essere superate impiegando una seconda coppia di altoparlanti distanziati in modo diverso e avendo una rete di modifica diversa dalla prima coppia; oppure può essere utilizzata una fila di altoparlanti con una rete composita, che modifica progressivamente le loro reti; oppure i due altoparlanti possono essere posizionati relativamente vicini tra loro.

In quest'ultima disposizione il senso della direzione della sorgente sonora apparente verrà trasmesso ad un ascoltatore per l'intera gamma di frequenze per le posizioni che si trovano tra i diffusori; ma se si desidera trasmettere l'impressione che la sorgente sonora si sia spostata in una posizione oltre lo spazio tra gli altoparlanti, le reti di modifica possono essere disposte in modo da invertire la fase di quell’altoparlanti da cui si desidera che appaia la sorgente, e questo sarà sufficiente per trasmettere l'impressione desiderata per i suoni a bassa frequenza. Con questa disposizione di altoparlanti vicini, tuttavia, non sarebbe possibile effettuare un'illusione simile in relazione alle alte frequenze.

Il sistema finora descritto, per ricevere le onde sonore, utilizza due microfoni omnidirezionali. Possono essere usati anche microfoni direzionali poco distanziati, le uscite vengono modificate come indicato in modo tale che le uscite relative degli altoparlanti siano controllate sia dalle differenze di fase che dalle differenze di ampiezza in uscita dal microfono. Tali microfoni direzionali possono essere, ma non necessariamente, del tipo a gradiente di pressione(velocity microphones), e preferibilmente dotati di elementi conduttori mobili così leggeri da muoversi sostanzialmente come l'aria circostante.

I microfoni direzionali (ad esempio i microfoni a nastro(moving strip microphones) sono molto adatti per qualsiasi sistema in relazione all'invenzione e oltre ad essere utilizzati con le disposizioni di circuito sopra descritte possono anche essere impiegati con varie alterazioni nei circuiti. Questi microfoni danno una risposta variabile come il coseno dell'angolo di incidenza del suono rispetto alla direzione di incidenza normale o ottimale, e hanno quindi il vantaggio che un certo grado di separazione dell'uscita dell'altoparlante può essere ottenuto senza conversione di fase o come le modifiche di rete.

Sono possibili tre disposizioni generali che utilizzano direzionali e in tutti i casi i microfoni sono posizionati il più vicino possibile anziché essere distanziati come orecchie artificiali, come nel caso dei microfoni omnidirezionali.

(1) Due microfoni direzionali sono posizionati uno con il suo asse di massima risposta rivolto direttamente nella direzione del centro della scena, e l'altro con il suo asse ad angolo retto rispetto a quella direzione. Entrambe le capsule sono allineate e disposte in modo tale che la linea sia verticale, mentre la sorgente sonora si sposta su un piano orizzontale. Un esecutore che parla dal centro della scena influenzerà solo il microfono frontale, ma se si sposta su un lato influenzerà entrambi i microfoni, mentre se si sposta sul lato opposto vengono fornite output simili ma la fase dei microfoni bidirezionali è vicina e non si riscontra alcuna differenza di fase tra di loro e se le loro uscite sono sommate e  sottratte dopo un'adeguata quantità di amplificazione, i due canali finali si differenziano in ampiezza in modo da pilotare gli altoparlanti ed ottenere l'effetto direzionale desiderato. Tale disposizione di somma e differenza differisce dalla rete di modifica impiegata con microfoni omnidirezionali in quanto il tipo di pressione fornisce differenze di fase (per cui viene determinata la direzione) che devono essere convertite, mentre con il microfono direzionale edge-on viene fornita un'uscita proporzionale alla direzione della fonte. Una disposizione di modifica adatta per questa forma dell'invenzione è mostrata in Figura 6. Questo è sostanzialmente identico a quello mostrato in Figura 5, tranne per il fatto che il condensatore shunt F e la resistenza K in serie, e la resistenza shunt I sono sostituiti dalle resistenze shunt L e M preferibilmente variabili come mostrato. Queste linee formano degli attenuatori artificiali e modificando la loro attenuazione relativa   vengono controllate le differenze di intensità nelle due linee corrispondenti a una data direzione del suono.

(2) Due microfoni direzionali, possono essere posizionati con i loro assi perpendicolari tra loro e ogni asse a 45 ° rispetto alla direzione del centro dello schermo. Questa disposizione è rappresentata in Figura 2 dove \textit{n} e \textit{o} rappresentano due microfoni direzionali uno sopra l'altro posizionati con gli assi perpendicolari l'uno all'altro e ad angoli uguali a 45° rispetto alla direzione del centro del campo da cui deve essere ricevuto il suono. In questo modo il movimento della sorgente sonora \textit{a} lateralmente in una posizione \textit{p} dal centro del campo provocherà delle onde sonore che colpiscono \textit{o} con un angolo più acuto di quanto colpiscono \textit{n} e quindi risulteranno delle differenze nell’output dei microfoni. Questi sono sufficientemente vicini per rendere trascurabili le differenze di fase del suono incidente e le ampiezze di uscita differiscono quindi approssimativamente in proporzione alla direzione del suono incidente. Possono quindi essere amplificati in modo simile e forniti direttamente agli altoparlanti a cui daranno le differenze di ampiezza corrette per l'effetto direzionali, a condizione che il rapporto tra le varie dimensioni della registrazione e la disposizione di riproduzione siano corretti. Se si desidera compensare eventuali differenze tra la disposizione degli altoparlanti(lay-outs), le uscite possono essere modificate da reti, nel modo descritto, opportunamente per aumentare o diminuire le differenze tra loro. Una disposizione come quella mostrata nella Figura 6 è adatta a questo scopo, e tale disposizione può ovviamente anche essere impiegata anche se la disposizione è corretta se si desidera per qualsiasi motivo controllare o modificare le differenze di ampiezza dell'altoparlante uscite.


 (3) Due microfoni possono essere disposti con i due assi che si trovano simmetricamente alla direzione del centro del campo e con un angolo tra loro di circa $\theta$ gradi, in modo che il suono di un esecutore al centro sottenda un angolo di $\frac{\theta}{2}$ gradi per ciascun microfono. Se $\theta$ è piccolo, è sufficiente un piccolo movimento dell’esecutore per essere catturato da un microfono (edge-on) un microfono e ridurne l'uscita a zero, mentre se $\theta$ è grande è necessario un grande movimento dell'esecutore. Rendendo $\theta$ regolabile, è possibile sistemare diverse disposizioni senza la modifica indicata al punto (2) e sarà chiaro anche che ciò fornisce un metodo di trasmissione, registrazione e riproduzione del suono direzionale che evita la necessità di combinare e separare i due canali.
 
Gli elementi del microfono in uno dei casi di cui sopra possono essere racchiusi in un unico involucro se lo si desidera per comodità, e possono anche essere posizionati in un unico sistema magnetico comune a entrambi.

Due microfoni direzionali allineati l'uno con l'altro e con i loro assi di massima risposta inclinati simmetricamente alla direzione della linea centrale della scena, possono, se posizionati uno sopra l'altro, essere impiegati anche per fornire significato sia verticale che orizzontale al movimento della sorgente sonora su un piano perpendicolare all'asse di massima risposta del sistema microfonico. Tale spostamento verticale fornisce differenze di fase alle uscite mentre lo spostamento laterale dà differenze di ampiezza modificando le reti, come descritto, e gli impulsi risultanti impiegati per far funzionare quattro o più altoparlanti distribuiti accanto allo schermo. La trasmissione in un tale sistema occupa solo due canali (uno per ciascun microfono) tuttavia ognuno di questi canali fornisce così quattro canali in tutto. Due canali, uno per ciascuna coppia parallela di questi canali divisi, sono collegati a una rete di modifica adattata per gestire le differenze di fase, e gli altri due canali, uno per ciascuna coppia, collegati a un'altra rete di modifica adattata per aumentare le differenze di intensità. Ogni rete di modifica opera una pluralità di altoparlanti che forniscono una sensazione direzionale in una direzione, e in questo modo si possono ottenere i sensi direzionali in due direzioni ad angolo retto. Si vedrà che in tale disposizione la trasmissione e/o la registrazione (che è l'operazione più estesa e difficile del sistema) può essere effettuata su solo due canali sebbene successivamente vengano ottenute sensazioni direzionali in due direzioni perpendicolari. Un effetto simile può essere ottenuto con una pluralità di microfoni omnidirezionali impiegando opportune modifiche prima della trasmissione.

Utilizzando una pluralità di microfoni è possibile ottenere un immagine sonora direzionale completa cioè con effetti direzionali sia verticali che orizzontali non limitandosi ad usare solo due microfoni. È possibile raccogliere, modificare e separare le uscite di una pluralità di microfoni per trasmettere gli impulsi con le opportune differenze ad una pluralità di altoparlanti. La caratteristica generale è che due canali di trasmissione, ricevendo ad esempio impulsi da due o più microfoni, comunicano impulsi che possono essere modificati e separati per fornire due sensi direzionali ad angolo retto l'uno con l'altro, quindi i suoni vengono fatti attraverso una pluralità di altoparlanti. Sarà inoltre chiaro che se la sorgente sonora si allontana da o verso i microfoni, l'intensità complessiva delle propagazioni combinate dei diffusori vari varierà e fornirà quindi l'indicazione della posizione della sorgente lungo quell'asse. In questo modo si ottiene così la posizione tridimensionale della sorgente.

Si vedrà che mentre con i microfoni omnidirezionali si preferisce trasmettere differenze di fase piuttosto che differenze di ampiezza e convertirle l'una dall'altra il più tardi possibile prima della riproduzione, con i microfoni direzionali è più conveniente trasmettere i due canali in fase ma a diverse ampiezze, l'unica modifica quindi necessaria è un aumento o una diminuzione delle differenze di ampiezza nel caso in cui la disposizione dell’impianto riproducente differisca dalla disposizione di registrazione o dovrebbero essere utilizzate più di due posizioni di altoparlanti.

Esiste un metodo semplice mediante il quale è possibile effettuare modifiche per aumentare o ridurre le differenze tra i canali se non è richiesta la conversione delle differenze di fase in differenze di ampiezza. Il metodo è particolarmente utile per il funzionamento di più di due altoparlanti, ed è anche utile per lavorare su impedenze elevate come l'impedenza di rete di una valvola termoionica. La disposizione è mostrata schematicamente nella Figura 7. Se la trasmissione viene effettuata sotto forma di due canali \textit{r s} di fase simile ma con ampiezze diverse, un'alterazione di queste differenze di ampiezza può essere effettuata collegando un filo di ciascun canale \textit{r} e \textit{s} insieme a \textit{t} e collegando un induttore \textit{u} tra gli altri due fili dei due canali. I canali in uscita \textit{v} e $\omega$ la cui differenza deve essere una modifica della differenza originale, sono collegati da un filo ciascuno al punto comune \textit{t} dei canali originali e dai loro altri fili alle prese lungo l’induttore \textit{u}. Se le differenze devono essere aumentate, le prese a cui sono collegati i canali di uscita si trovano al di fuori delle prese a cui sono collegati i canali di ingresso, in modo che lo starter funzioni in effetti come un autotrasformatore che amplifica le tensioni di differenza. Allo stesso modo, per una riduzione delle differenze, i canali di uscita vengono toccati in modo intermittente tra i due canali di ingresso. Modifiche di questa disposizione in cui i dispositivi sono bilanciati rispetto alla terra, ecc. possono essere utilizzate, ma il vantaggio principale è che la modifica viene variata completamente alterando le prese lungo un trasformatore o un induttore, e che non comporta alcuna grande perdita di potenza.

Questa disposizione di un induttore o di un trasformatore si adatta perfettamente al funzionamento di numerosi altoparlanti per la riproduzione binaurale. In questo caso, le due uscite dalle valvole di potenza sono montate su un induttore del tipo \textit{u} lungo il quale vengono toccati gli altoparlanti. La posizione delle prese degli altoparlanti può essere regolata per adattarsi alle loro posizioni relative e si può disporre che le valvole funzionino nelle loro migliori impedenze. I trasformatori possono essere utilizzati per garantire che gli altoparlanti prendano la loro frazione corretta dell’uscita.

Quando, come riportato sopra, si suggerisce di registrare per una riproduzione cinematografica, deve essere chiaro che l'invenzione non è limitata a quel mezzo poiché la registrazione può essere effettuata su dischi o cilindri di materiale adatto. Nel realizzare l'invenzione in questo modo i due canali possono essere registrati in solchi separati ma è preferibile che siano registrati nello stesso solco con incisioni sia laterali che verticali. Ai fini televisivi è stata proposta una registrazione su un disco di cera con un incisione sia laterale che verticale e la registrazione appare come un V tagliata alla base dei solchi verticale e laterale o viceversa.. Tali registrazioni sembrano inadatte a registrazioni sonore separate e distinte poiché si verificherebbe indubbiamente un effetto cross-talk considerevole tra le due registrazioni. Possono tuttavia essere usati per due canali del tipo previsto nella presente invenzione, l'uno essendo solo leggermente diverso dall'altro, poiché in questo caso una certa quantità di interferenza è irrilevante o può essere consentita. Inoltre, le registrazioni proposte ora si distinguono da quelle precedenti in quanto entrambi i canali possono essere registrati come incisioni separate in un unico solco e possono essere registrati da un singolo strumento di registrazione (sia di ferro mobile che di tipo a bobina mobile) e possono essere riprodotti da singolo riproduttore o pick-up.

Se i due canali in fase di registrazione vengono acquisiti direttamente da due microfoni o intendono funzionare senza essere modificati direttamente in due altoparlanti, quindi con intensità e qualità simili a quelle dei suoni originali, è preferibile non incidere il solco con un taglio laterale e verticale, ma come due tracce i cui assi di movimento si trovano a 45° rispetto alla superficie della cera, o ad un altro angolo dipendente dalle relative intensità disponibili rispettivamente dall’incisione laterale e verticale. Se, tuttavia, i due canali registrati sono come i canali di somma e differenza, è preferibile separarli completamente nelle incisioni laterale e verticale, vale a dire per rendere gli assi di registrazione normali e tangenti alla superficie della cera.
 
Il risultato nei due casi sopra suggeriti è molto simile poiché i canali registrati a 45° sulla superficie della cera danno la loro somma e differenza come effettive ampiezze laterale e verticale.
 
Si apprezzerà che un disco, inciso combinando tagli laterali e verticali, può essere riprodotto se desiderato con due incisioni in direzione obliqua, il principio di base è che il solco ha ampiezza in qualsiasi direzione in un piano ad angolo retto rispetto alla direzione del movimento della cera e le direzioni di registrazione e riproduzione possono essere scelte come qualsiasi coppia di assi, non necessariamente ad angolo retto, su questo piano.

Per questo tipo di registrazioni è consigliabile utilizzare un materiale diverso da quello utilizzato ora per le laterali come l’acetato di cellulosa.

La sezione del solco è preferibilmente adattata per lavorare con uno zaffiro e ha un angolo sufficientemente fine da dare allo zaffiro un controllo laterale e verticale.
 
Il registratore in grado di incidere entrambi i canali sullo stesso solco con lo stesso strumento può assumere varie forme, la caratteristica sottostante è che uno stilo leggero viene tirato in due direzioni ad angolo retto l'uno rispetto all'altro e ciascuno preferibilmente a 45° rispetto alla superficie della cera.

La figura 8 mostra lo schema di un registratore di questo tipo adatto a produrre dischi con incisioni complesse. 1 e 2 rappresentano gli elementi guida di due registratori normalmente utilizzati per le incisioni laterali dei dischi. Questi guidano i bracci 3 e 4 attorno agli assi ad angolo retto rispetto al piano della carta all’interno 1 e 2. Le estremità di questi bracci sono collegate dai legamenti 5 e 6 all'estremità di una lamella 7 che si estende all'indietro lungo un asse perpendicolare alla carta a supporti non mostrati. Questa lamella porta uno zaffiro da taglio 8. I movimenti dei bracci di registrazione 3 e 4 producono movimenti alla fine della lamella 7. Quindi i movimenti guidati da 1 faranno sì che la lamella 7 si sposti lungo un asse di circa 45° rispetto alla verticale che sale da sinistra a destra attraverso la figura. Analogamente, mentre quelli da 2 produrranno il movimento della lamella 7 con un asse ad angolo retto rispetto al precedente asse, influenzando entrambi i movimenti della lamella.

Un’altro tipo di registratore è mostrato in Figura 9, che rappresenta un registratore di ferro in movimento, può consistere in una lamella corta 9 montata vicino e parallela alla pista di cera e recante lo zaffiro da taglio 8. Questa lamella 9 può estendersi all'indietro perpendicolarmente alla carta per supporti (non mostrati) che si uniscono alla sommità di un sistema a palo laminato 10 per completare un sistema magnetico polarizzante con esso. I due bracci laminati del pezzo polare 10 si estendono verso il basso verso l'estremità libera della lamella 9. Questi bracci formano due poli adiacenti ad una porzione quadrata della canna alla sua estremità libera, ciascun polo è adattato a tirare la canna in una direzione a 45° rispetto alla superficie della cera. La canna può essere opportunamente smorzata, ad es. da una linea di gomma e hanno una frequenza di risonanza nella parte superiore o inferiore del campo di lavoro. I due pezzi polari possono essere avvolti con bobine e l'eccitazione di uno di questi sposta lo zaffiro in una direzione verso l'alto a 45° rispetto alla superficie della cera. I terminali 15 di un canale sono collegati all'avvolgimento principale 12 e all'avvolgimento di compensazione 11. Il terminale 16 dell'altro canale è collegato all'avvolgimento principale 14 e all'avvolgimento di compensazione 13. La corrente in entrambi i canali tirerà la lamella verso il polo portante l'avvolgimento principale lo scopo dell'avvolgimento di compensazione è quello di impedire il movimento della lamella lontano dall'altro polo a causa del flusso preso da questo polo dall'avvolgimento principale. Con l'avvolgimento mostrato, le guide in entrambi i canali faranno sì che la lamella tagli una pista a circa 45 ° in verticale. Tramite un'adeguata riorganizzazione degli avvolgimenti o mediante un'adeguata connessione del trasformatore tra i canali e i terminali del registratore, come mostrato, è possibile ottenere un movimento su un qualsiasi altro asse. Quindi, ad esempio, è possibile creare un movimento mediante torsione della sua canna di supporto e un altro mediante la sua flessione.

Un design alternativo della bobina mobile che può impiegare uno smorzamento elettromagnetico può consistere in un elemento mobile a forma di T come mostrato in Figura 10. Lo zaffiro del registratore 8 è supportato su un elemento T leggero 17, che è supportato a 18 con mezzi elastici in modo tale che possa ruotare attorno a questo punto, e può anche traslare verticalmente, sebbene sia resistente ai movimenti orizzontali nel piano. Il dispositivo è azionato da bobine mobili, ad es. bobbina, 19 e 20 che sono liberamente posizionate e immerse nel campo magnetico costante fornito in spazi anulari in un sistema magnetico, non mostrato. La corrente in una delle bobine mobili tende sia a ruotare sia a traslare il dispositivo in modo tale che lo zaffiro 8 si sposti lungo un asse di circa 45 ° rispetto alla verticale. Il movimento di questo dispositivo può essere smorzato ed equalizzato secondo le linee descritte nella specifica del brevetto britannico n. 350.998. Può essere quindi ottenuto un movimento su un qualsiasi asse mediante un'adeguata interconnessione delle due bobine di comando. Un tale movimento ha preferibilmente la stessa frequenza naturale sia per rotazione che per traslazione. Inoltre, la distribuzione della massa è preferibilmente tale che una piccola forza istantanea applicata su una bobina non produce alcun movimento sull’altra.

La Figura 11 mostra un'altra forma di registratore simile in linea di principio a quella mostrata nella Figura 10, tranne per il fatto che viene impiegato un motore in ferro mobile. L'elemento 17 che si muove attorno all'asse 18 è costruito in materiale magnetico o ha una porzione superiore magnetica. L'elemento a forma di "E" 21 è polarizzato o essendo parzialmente e permanentemente magnetizzato, o avendo un avvolgimento magnetizzante su di esso, in modo che il polo centrale sia di polarità opposta ai due poli esterni. Gli “speech windings” sui poli esterni vengono portati ai terminali 15 e 16 a cui sono collegati i due canali.

In tutti i dispositivi sopra descritti, gli angoli degli assi che definiscono i movimenti dello zaffiro possono essere modificati collegando opportunamente gli “speech windings”; per esempio, gli assi che sono normalmente inclinati di 45 ° rispetto alla superficie della cera possono essere convertiti in puri assi di incisione verticali o laterali, disponendo che gli “speech windings” siano in serie a favore di un canale e opposti per l'altro canale. Allo stesso modo qualsiasi conversione degli assi può essere effettuata combinando opportunamente i canali attraverso trasformatori.

Nel progettare un pick-up elettrico per riprodurre entrambi i canali, si deve fare attenzione a mantenere l'inerzia il più bassa possibile, e con questo in mente può essere impiegata una replica molto leggera di uno qualsiasi dei registratori sopra descritti. Preferibilmente, un sistema mobile sotto forma di una T che segue le linee del registratore di ferro mobile mostrato in Figura 11. Poiché la frequenza di risonanza fondamentale di un pick-up sembra non avere alcuna importanza critica per quanto riguarda le sue caratteristiche, potrebbe non essere necessario adattare la frequenza di risonanza nelle due modalità allo stesso valore, il che semplificherebbe la progettazione. Regolazioni della sensibilità nelle due modalità possono essere effettuate collegando opportunamente bobine avvolte sui due bracci dei circuiti magnetici. Come nel progetto del registratore, la distribuzione della massa nel riproduttore è preferibilmente tale che le forze che producono movimento in una direzione (ad es. movimenti laterali) lo lasciano sostanzialmente indisturbato nella sua riproduzione da movimenti in un'altra direzione (ad es. verticali).

Un buon effetto binaurale può essere ottenuto dando significato direzionale solo a una gamma limitata di frequenze. Ad esempio, sebbene una buona riproduzione richieda la trasmissione di tutte le frequenze fino a 10.000 c.p.s. tuttavia si ottiene un buon effetto direzionale da frequenze fino a 3.000 c.p.s. Ciò aiuterebbe la registrazione del disco degli impulsi binaurali poiché il taglio laterale che rappresenta la somma dei due canali agli altoparlanti potrebbe avere una gamma di frequenza che si estende a 10.000 c.p.s. mentre il taglio verticale deve trasmettere frequenze non superiori a 3000 c.p.s. Ciò semplificherebbe notevolmente il design dei registratori e i pick-up in quanto le basse inerzie sarebbero necessarie solo per le incisioni laterali e il design sarebbe quindi notevolmente semplificato.

Queste frequenze sono fornite a puro titolo di esempio e non sono necessariamente le frequenze ottimali per la progettazione di questo sistema, che sarà determinato da altre considerazioni.

Nel trasmettere i due canali indicati nei vari sistemi sopra descritti, invece di utilizzare la trasmissione di linea, può essere impiegata la trasmissione radio. Ciascun canale può essere trasmesso separatamente o preferibilmente i due canali possono essere inviati come diverse modulazioni della stessa onda portante. Pertanto un canale può essere trasmesso come modulazione di ampiezza e l'altro come modulazione di fase o frequenza della stessa onda portante. In alternativa, i due canali possono essere trasmessi come modulazioni di ampiezza di onde portanti diverse che sono sfasate di 90 °, le due onde vengono irradiate dalla stessa antenna in combinazione come una onda singola. Sono noti vari sistemi per la trasmissione e la ricezione di segnali radio bidirezionali lungo queste linee e una qualsiasi di tali o simili disposizioni può essere usata in connessione con l'invenzione qui descritta secondo la sua applicabilità o convenienza nelle varie circostanze. Con un tale sistema di radiazione bidirezionale, è possibile eseguire uno dei processi di somma e differenza nel collegamento radio, ad esempio con la demodulazione all'estremità di ricezione con due onde portanti fuori fase di 90 ° , le cui onde portanti sono sfasate di 45 ° rispetto alle portanti originali, i canali a bassa frequenza risultanti sono le somme e le differenze dei canali a bassa frequenza originali sul trasmettitore.
 
Il sistema qui descritto, pur essendo particolarmente applicabile ai film, non si limita a tale uso. Può essere impiegato nella registrazione del suono in modo del tutto indipendente da qualsiasi effetto di immagine e in questo modo (così come nel lavoro cinematografico) l'effetto binaurale introdotto può servire a migliorare le proprietà acustiche degli studi di registrazione e salvare qualsiasi drastico trattamento acustico della stessa fornendo al contempo registrazioni molto più realistiche e soddisfacenti per la riproduzione. Inoltre, il sistema può essere chiaramente impiegato quando le uscite del microfono sono direttamente connesse agli altoparlanti invece di essere prima registrate, e una tale disposizione può ad esempio essere impiegata in sistemi di diffusione sonora in cui sono desiderati effetti sonori direzionali. In generale, l'invenzione è applicabile in tutti i casi in cui si desidera dare effetti direzionali al suono emesso. Anche in tutti i casi, sia quando gli impulsi vengono inviati agli altoparlanti senza registrazione sia quando vengono registrati per la successiva riproduzione, la modifica totale e/o l'interazione dei canali può essere realizzata in più di uno stadio. Ad esempio, usando microfoni omnidirezionali, le differenze di fase a bassa frequenza possono essere aumentate, le differenze di fase a media frequenza convertite in differenze di ampiezza e le differenze di ampiezza ad alta frequenza aumentate in un primo stadio di modifica; le differenze di fase a bassa frequenza possono quindi essere convertite in differenze di ampiezza in uno stadio successivo di modifica. Una o entrambe queste fasi possono verificarsi prima o dopo la registrazione del suono. In questo modo le piccolissime differenze di fase a bassa frequenza vengono aumentate prima di essere amplificate, evitando così problemi dovuti a piccoli spostamenti di fase a bassa frequenza negli amplificatori.

Inoltre, i vari dispositivi impiegati per attuare l'invenzione devono essere intesi come non limitati al loro uso con gli altri dispositivi nei sistemi qui descritti, poiché chiaramente molte parti, come ad esempio il registratore a doppio traccia preparato con una singola incisione e il microfono di rilevamento della direzione multi-strip sono chiaramente di grande utilità in tali sistemi separatamente l'uno dall'altro. Tali usi nei sistemi binaurali come qui descritti rientrano nell'ambito della presente invenzione.

L'invenzione non è limitata esclusivamente ai dettagli delle disposizioni delle forme sopra descritta poiché possono essere introdotte varie modifiche al fine di mettere in atto l'invenzione in condizioni e requisiti diversi che devono essere soddisfatti senza allontanarsi in alcun modo dal suo ambito.

Avendo ora in particolare descritto e accertato la natura della nostra invenzione e in che modo deve essere eseguita la stessa, dichiariamo che ciò che rivendichiamo è:

1. Un sistema di trasmissione del suono dove il suono è ripreso da una pluralità di microfoni e riprodotto da una pluralità di altoparlanti, comprensivo di due o più microfoni sensibili alla direzione e/o un sistema di elementi nel circuito di trasmissione o circuiti per cui il volume relativo degli altoparlanti è reso dipendente dalla direzione da cui arrivano i suoni ai microfoni.

2 . Un sistema di trasmissione del suono ricevuto da due o più microfoni, dove le basse frequenze che differenziano per la fase della pressione sonora sui microfoni è riprodotta come differenza di volume dagli altoparlanti.

3. Un sistema di trasmissione del suono , in cui il suono originale è ripreso da due o più microfoni direzionali la cui sensibilità varia con la direzione del suono incidente, ed in cui la dipendenza delle risposte dei microfoni relativi ad un onda sonora incidente viene utilizzata per controllare i volumi relativi di un suono emesso da due o più altoparlanti.

4. Un sistema di trasmissione dove due o più microfoni sono usati per catturare il suono originale, e sia la fase che il volume del segnale in uscita dai due microfoni è utilizzato per controllare, relativamente alla direzione di incidenza del suono originale, il relativo volume di uscita di due o più altoparlanti.

5. Un sistema in accordo con la dichiarazione 1 e 4 in cui due o più canali sono combinati e separati in altri canali così che i canali risultanti, anche se non simili ai precedenti canali, risultano delle loro modifiche che trasmettono la stessa informazione direzionale in un altra forma.

6. Un sistema di trasmissione del suono, dove gli impulsi da due microfoni trasmessi sopra canali individuali sono adattati per interagire in modo da consistere rispettivamente in metà della somma e metà della differenza degli originali, vengono quindi trasmessi e successivamente modificati per controllare il volume relativo degli altoparlanti che devono riprodurre i suoni.

7. Un sistema in accordo con la dichiarazione 6 in cui dopo la modifica delle due serie di impulsi sono trattati attraverso una ripetizione del processo di somma e differenza inizialmente effettuato.

8.  Un sistema in accordo con la dichiarazione 6 o 7 dove dopo che gli impulsi iniziali sono stati trasformati in somma e differenza, la modifica degli impulsi viene effettuata in ciascun canale di somma e differenza mediante reti di attenuazione e/o disposizioni di modifica di fase.

9. Un sistema in accordo con la dichiarazione 8 dove la modifica degli impulsi è effettuata da semplici elementi shunt, ad esempio resistenze e/o condensatori, che possono variare il valore. 

10.  Un sistema in accordo con tutte le dichiarazioni dalla 5 alla 9 dove parte della differenza di fase del rage di frequenze è convertita in differenza di ampiezze nei canali risultanti.

11. Un sistema in accordo con tutte le dichiarazioni dalla 5 alla 10 in cui su tutta o parte del range di frequenze le differenze trai canali sono aumentate o ridotte.

12. Un sistema in accordo con tutte le precedenti dichiarazioni in cui due microfoni sono separati da una piccola distanza approssimatamente uguale alle orecchie umane.

13. Un sistema in accordo alla dichiarazione 12 comprendente due microfoni omnidirezionali tra i quali è prevista un divisore.

14. Un sistema in accordo con tutte le dichiarazioni 2 o dalla 4 alla 12 che comprendono microfoni direzionali.

15. Un sistema di trasmissione del suono dove i suono è ripreso da due microfoni direzionali che sono spaziati e/o con i loro assi di massima sensibilità orientati uno rispetto all’altro e alla sorgente sonora e il volume relativo degli altoparlanti che riproducono gli impulsi è controllato dalla direzione da cui il suono raggiunge i microfoni.

16. Un sistema come dichiarato dalla 1 alla 11, 14 o 15 comprendente due microfoni direzionali posti in vicina giustapposizione con i loro assi di massima sensibilità puntanti in diverse direzioni.

17. Un sistema come dichiarato dalla 1 alla 12 o dalla 14 alla 16 comprendente microfoni direzionali o elementi microfonici nella forma da muoversi sostanzialmente come l’aria circostante.

18. Un sistema come dichiarato in tutti i punti da 1 a 2 o da 14 a 17 dove una pluralità, ad esempio due, di elementi microfonici direzionali sono costruiti in un singolo contenitore con un sistema magnetico comune o sistemi separati.
 
19. Un sistema come dichiarato nei punti da 1 a 12 o da 14 a 18 comprendenti due microfoni direzionali, uno perpendicolare alla direzione del centro del campo sonoro e un altro posto longitudinalmente alla linea e ad angolo retto rispetto al piano in cui il suono si muove.

20. Un sistema come dichiarato in tutti i punti da 1 a 12 o da 14 a 18 dove due microfoni direzionali sono posti a 45° rispetto a direzione del centro del campo sonoro.

21. Un sistema come dichiarato nei punti da 1 a 12, da 14 a 18 o 20 che comprende due microfoni direzionali in cui l’angolo fra di loro (quindi l’angolo di ciascuno rispetto al centro del campo sonoro) è modificabile.

22. Un sistema come dichiarato nei punti da 15 a 21 dove microfoni sensibili alla direzione sono disposti o diretti in modo da fornire impulsi per cui si ottengono i livelli relativi desiderati dei diffusori, questi impulsi vengono trasmessi ai diffusori senza modifica o interazione.

23. Un sistema come dichiarato nei punti 15-21 dove, gli impulsi creati da un microfono direzionale sono modificati (ad esempio da una rete di attenuatori nei canali somma e differenza) prima di essere riprodotti dagli altoparlanti.

24. Un sistema come dichiarato nel punto 23 dove le modifiche sono effettuate da una semplice bobbina inserita tra i due microfoni e due altoparlanti, quattro cavi (uno per ogni microfono e ogni altoparlante) sono collegati insieme mentre l'altra parte da ogni membro è connesso in modo mobile alle prese della bobbina. 

25. Un sistema come quello dichiarato nei punti 1-21, 28 o 24 dove la modifica dell'impulso è fatta in due o più stadi.

26. A system aa claimed in any preceding claim wherein the transmitted impulses are photographically recorded upon separate film sound tracks, preferably adjacent to one another, either track being either of the variable width, or variable density form.

27. A system as claimed in any of Claims1-25 wherein a record of two sets of impulses is located upon a single film track in the form of a combined variable width and variable density recording. 

28. A system as claimed in any of Claims 1-25 wherein the impulses are recorded upon discs or cylinders of wax or like suitable material.

29. A system as claimed in any of Claims 1-25 or 28 wherein the two sets of impulses are recorded upon the same cylinder or disc.

30. A system as claimed in Claim 29 wherein two sets of impulses are mechanically recorded in the same groove.

31. A system as claimed in Claim 29 or 30 wherein one record is a lateral cut and the other a hill-and-dale cut in a single groove.

32. A system as claimed in any of Claims 28-31 wherein the recordings are effected simultaneously.
 
33. A system as claimed in any of Claims 28-32 wherein the recordings are effected by a single cutting tool.

34. A system as claimed in Claim 33 where in the cutting tool is capable of controlled movement in all directions in a plane perpendicular to the direction of movement of the wax. 

35. A system as claimed in Claim 33 or 34 wherein the cut of the recording tool is in form a combination of lateral and hill-and-dale cuts, or equivalent to that form.

36. A system as claimed in any of Claims 28-35, wherein one channel is recorded as a cut in a direction at an angle to the normal to the wax and the other channel is recorded as a cut at the same angle to the normal to the wax but on the other hand relative to the groove.

37. A system as claimed in any of Claims 26-36, wherein the desired modification of the two channels is wholly effected either before recording or after reproduction from the record, or is partially effected in each stage. 

38. A sound reproducing system wherein the sounds are reproduced without modification from one or more sound records prepared by a system according to any preceding claim. 

39. Microphone arrangements for a system according to any preceding claim comprising a plurality of directionally sensitive elements arranged with a common magnetic system, or separate magnetic systems, in a common casing or container, the axes of maximum sensitivity of the elements being arranged at an angle to one another, the elements being connected to separate transmission channels whereby the impulses are separately transmitted from the elements.
 
40. Microphone arrangements as claimed in Claim 39, wherein the angle between the elements is adjustable.

41. A system according to any of Claims 28-37, embodying a sound recorder comprising an operating movement adapted to respond to both channels of impulses and to cut records of both simultaneously.

42. A system embodying a sound recorder as claimed in Claim 41 comprising a single cutting tool adapted to operate in a single groove.

43. A system embodying a sound recorder as claimed in Claim 41 or 42 adapted to effect a different kind of recording cut (e.g. lateral or hill-and-dale) for each of the channels. 

44. A system embodying a sound recorder as claimed in (Claim 41 or 42 adapted to respond in similar manner to, and effect cuts of similar form for, each channel, but in opposite sense relative to the groove. 

45. A system embodying a sound recorder as claimed in any of Claims 41-44 comprising a cutting tool capable of controlled movement in all directions in a plane perpendicular to the direction of movement of the wax. 

46. A system embodying a sound recorder as claimed in any of Claims 41-45 comprising a cutting tool the movements of which in two directions, perpendicular or at an angle to one another, are separately controlled by the driving arms of separate recording elements.

47. A system embodying a sound recorder as claimed in any of Claims 41-45 comprising a cutting tool carried upon a flexible reed adapted to be moved in either of two directions, perpendicular or at an angle to one another; or to be subjected to a resultant movements equivalent to the combination of such motions in both directions.

48.  A system embodying a sound recorder as claimed in Claims 47 wherein movements of the reed are effected by electromagnetic forces imposed by adjacent poles of a co-operating magnetic system excited by the impulses to be recorded.

49. A system embodying a sound recorder as claimed in Claim 48 wherein compensating coils are wound on the magnetic system tn addition to the exciting speech coils in order to neutralise effects upon one pole of impulses in the speech coil of the other pole.

50. A system embodying a sound recorder as claimed in any of Claims 41-45 comprising a cutting tool assembly adapted to have one movement by torsion of its supporting reed and another by flexure thereof.

51. A system embodying a sound recorder as claimed in Claim 50 wherein the cutting cool assembly is driven by moving coil drives, comprising speech coils attached thereto freely immersed in  a steady magnetic field. 

52. A reproducer for a transmitting system as claimed in any of Claims 1-38, comprising elements of small weight and inertia but otherwise substantially identical in form and arrangement with those of a recorder as claimed in any of Claims 41-51, the cutting tool therein being replaced by a stylus whereby electrical vibrations are produced in the magnetic windings by vibrations imparted to the movable reed or like movable armature. 

53. A system as claimed in Claim 41 embodying a sound recording or sound reproducing device substantially as represented in Figure 9.
 
54. A system as claimed in Claim 41 embodying a sound recording or sound reproducing device substantially as represented in Figure 10. 

55. A system as claimed in Claim 41 embodying a sound recording or sound reproducing device substantially as represented in Figure 11. 

56. A sound record prepared by a system as claimed in any of Claims 1-38 or 44-51. 

57. A sound record comprising in one groove two substantially separate records of sound which emanate from the same source, Which sounds are picked up by directionally sensitive devices and/or are subjected to modifications by elements in the recording circuit, in such a manner that when the records are reproduced, one in one loud speaker and the other in another loud speaker, the intensities of the sounds simultaneously propagated convey in combination a true binaural effect to the listener. 

58. A sound record as claimed in Claim 57 comprising separate cuts of different form, or of same forms along any pair of axes in a plane perpendicular to the direction of movement of the was, for the separate recordings. 

59. A sound reproducing device adapted to reproduce sounds from motions in one direction (e.g. lateral movements) on a sound record prepared according to any of Claims 43-51, while remaining substantially undisturbed in its reproduction by motions in another direction (e.g. hill- and-dale).

60. A system as claimed in any of Claims 1-39 wherein the impulses are transmitted by radio telephony.

61. A system as claimed in Claim 60 wherein transmission is effected by the duplex modulation of a single carrier wave, or by radiation of a single wave formed of separately modulated components.

62. A system as claimed in any of Claims 1-39, 60 or 61 in combination with means for the photographic recording, or for the reproduction of pictures. 

63. A sound and picture reproducing system as claimed in Claim 62 wherein the relative volumes of the reproducing loud speakers are so controlled as to provide apparent location of the sound origin in coincidence with the optical location of the image from which the sound is supposed to emanate.

64. A system as claimed in Claim 62 or 63 wherein the relative modifications of the two channels is determined by the dimensions and lay-out of scene to be recorded, and/or the theatre in which reproduction is presented.

65. A system as claimed in Claim 62, 63 or 64 wherein the relative values of the modifying networks in the two channels is defined by the formula :
\[
K = \frac{2\nu}{j\omega} * \frac{y}{\theta k} * \frac{s}{x}
\]
where the symbols have the meanings defined herein. 

66. A system as claimed in any of Claims 1-39 or 60-65, or plurality of such systems in combination as a single system, adapted to provide a full directional significance to sounds emitted by a source movable in any direction in a plane-perpendicular to the axis of maximum response of the microphone system.

67. A system as claimed in Claim 66 wherein the total sound emission of all loud speakers determines the position of the sound source along the said axis of maximum response, so that full three-dimensional acoustic location of the sound source is obtained. 

68. Systems of sound transmission substantially as described herein, with reference to the accompanying drawings. 

69. Means for the transmission, recording and reproducing of sound substantially as described herein with reference to the accompanying drawings. 

70. Systems for the transmission, recording and reproduction of combined picture and sound effects substantially as described herein with reference to the accompanying drawings. 

\begin{flushright}

9 - 11 - 1932. 
MARKS & CLERI. 
\end{flushright}




 



 
 

 
 



 

 



 
 
 






 




 










\end{multicols*}

\end{document}